\onecolumn
\addchap{Zur Reihe\vspace{-1em}}
Unter den großen Komponisten des 18. Jahrhunderts im deutschsprachigen Raum nimmt Wolfgang Amadé \index[personen]{Mozart, Wolfgang Amadé (1756--1791)}Mozart in vielerlei Hinsicht eine Sonderstellung ein: Für keinen anderen Komponisten dieser Zeit ist eine ähnlich reiche und lebendige Familienkorrespondenz erhalten geblieben; kein anderer Musiker hat wie er schon als Kind internationale Aufmerksamkeit, die bis in die Tagespresse durchschlug, erregt; sein Ruhm endete nicht – wie bis dahin üblich – mit seinem Tod, sondern hält ungebrochen bis heute an. Das vielseitige Schaffen \index[personen]{Mozart, Wolfgang Amadé (1756--1791)}Mozarts, das sich auf alle Gattungen der mitteleuropäischen Musik seiner Zeit erstreckte, gab schon wenige Jahre nach seinem Tod den entscheidenden Anstoß für die Idee einer Musiker-Gesamtausgabe.

%\index[personen]{Mozart, Wolfgang Amadé (1756--1791)}Mozart kann aber auch aufgrund des reichen Materials, das in fast 200 Jahren der wissenschaftlichen Auseinandersetzung zusammengetragen wurde, als exemplarisch für einen entscheidenden Umbruch in der Geschichte der europäischen Musik angesehen werden: In dieser Zeit vollzog sich vielerorts der Wandel vom Beruf des festangestellten Musikers in kirchlichen oder höfischen Diensten zum ‚freien‘ Künstler, der mit einer Neugewichtung des Verhältnisses zwischen Auftragskomposition und Komposition aus eigenem Antrieb einhergeht.

Die Recherchemöglichkeiten, die das Internet bietet, haben dazu geführt, dass in jüngster Zeit eine Vielzahl an neuen Dokumenten zur Lebens- und Wirkungsgeschichte Wolfgang Amadé \index[personen]{Mozart, Wolfgang Amadé (1756--1791)}Mozarts und seiner Familie bekannt geworden ist. Viele dieser Treffer sind Zufallsfunde, die ahnen lassen, wie lohnend systematische Untersuchungen sein können, die aber nur im Rahmen von Sonderprojekten mit institutioneller Unterstützung mit Aussicht auf Erfolg durchgeführt werden können. Das Internet bietet zudem auch neue Publikationsmöglichkeiten. Diese sind hilfreich, da neue Erkenntnisse ohne Zeitverzug und Kosten von jedermann publiziert werden können; die Kehrseite der Medaille zeigt sich aber bereits jetzt, da keineswegs befriedigend geklärt ist, in welchen Formaten und in welcher Form Erkenntnisse der Allgemeinheit dauerhaft zur Verfügung gestellt werden können.

Die Internationale Stiftung Mozarteum, Salzburg und das Archiv der Erzdiözese Salzburg, die gemeinsam mit ihrer Vorgängereinrichtung \textit{Dommusikverein und Mozarteum} von 1841 auf eine 175-jährige Geschichte der Mozart-Pflege zurückblicken, sowie der Carus-Verlag Stuttgart sind sich einig, dass das Buch auch bei datenbankgestützten Projekten auf lange Zeit ein adäquates Mittel bleibt, um Erkenntnisse zu publizieren, von denen zu erwarten steht, dass sie dauerhaft für die Mozart-Forschung von Interesse sind. Die Reihe \textit{Beiträge zur Mozart-Dokumentation} soll neben Dokumentensammlungen auch Kataloge einschließen.

Als Band 3 der Reihe erscheint nun ein besonders wichtiger Baustein zur \index[personen]{Mozart, Wolfgang Amadé (1756--1791)}Mozart-Pflege im 19. Jahrhundert, ein Katalog derjenigen musikalischen Werke, die aus dem Besitz von \index[personen]{Mozart, Carl Thomas (1784--1858)}Carl Thomas und Franz Xaver Wolfgang \index[personen]{Mozart, Franz Xaver Wolfgang (1791--1844)}Mozart zwischen 1844 und 1860 an den Dommusikverein und Mozarteum in Salzburg gelangt sind. Eine entscheidende Rolle kommt dabei neben den beiden Söhnen von \index[personen]{Mozart, Wolfgang Amadé (1756--1791)}Wolfgang Amadé und Constanze \index[personen]{Nissen, Constanze (1762--1842)}Mozart der Universalerbin Franz Xaver Wolfgang \index[personen]{Mozart, Franz Xaver Wolfgang (1791--1844)}Mozarts, Josephine \index[personen]{Baroni-Cavalcabò, Josephine von (1787--1860)}Baroni-Cavalcabò, zu, die dessen Vermächtnis, obwohl es nur mündlich hinterlegt war, zugunsten des \textit{Dommusikverein und Mozarteums} umsetzte und dem Verein auch ihre eigene Notensammlung schenkte.

Die zu unterschiedlichen Zeitpunkten nach Salzburg gelangten Teilsammlungen werden seither gemeinsam als \glqq Mozart-Nachlass\grqq{} bezeichnet. Dies widerspricht auf den ersten Blick unseren intuitiven Vorstellungen; denn heute denken wir bei dieser Bezeichnung voreilig meist nur an den musikalischen Nachlass Wolfgang Amadé \index[personen]{Mozart, Wolfgang Amadé (1756--1791)}Mozarts bei seinem Tod am 5. Dezember 1791. Umso wichtiger ist es daher, sich die historischen Bedingungen, unter denen diese einzigartigen Schenkungen erfolgten, zu verdeutlichen. Der \glqq echte (Teil-)Nachlass\grqq{} Franz Xaver Wolfgang \index[personen]{Mozart, Franz Xaver Wolfgang (1791--1844)}Mozarts, den der \textit{Dommusikverein und Mozarteum} 1844 entgegennehmen konnte und auf den die Bezeichnung ursprünglich abzielte, bildete nur den Anfang der wichtigsten Phase der Übergabe wichtiger Mozartiana aus Familienbesitz, wodurch das Selbstverständnis des \textit{Dommusikverein und Mozarteums} und seiner Nachfolgeorganistionen als \glqq Erbe\grqq{} der Familie Mozart begründet wird. Auch wenn im hier vorgelegten Katalog nur die Musikalien und theoretischen Werke im Detail beschrieben sind, wird deutlich, dass der \glqq Mozart-Nachlass\grqq{} ganz disparate Dinge – Musikalien (darunter zahlreiche Autographe von Wolfgang \index[personen]{Mozart, Wolfgang Amadé (1756--1791)}Amadé und Franz Xaver Wolfgang \index[personen]{Mozart, Franz Xaver Wolfgang (1791--1844)}Mozart), theoretische Werke, Originalbriefe, Dokumente, Bilder, Originalinstrumente und andere \glqq Effekten\grqq{} – umfasste.

Obwohl der \glqq Mozart-Nachlass\grqq{} in einem historischen Repertorium verzeichnet wurde, war die Aufgabe, ihn heute Stück für Stück zu beschreiben, weit schwieriger, als es auf den ersten Blick erscheinen mag. Denn einerseits ist der Bestand seit der 1880/81 vollzogenen Spaltung in den Dommusikverein und die \glqq Stiftung\grqq{} Mozarteum auf zwei Salzburger Institutionen verteilt, wobei dort viele der Quellen aus praktischen Erwägungen heraus neu signiert und damit aus dem ursprünglichen Zusammenhang gerissen wurden. Einzelne Stücke sind verloren gegangen; besonders schwer wiegt ein großangelegter Diebstahl in den 1960er-Jahren aus dem Salzburger Dom, wonach nur ein Teil der entwendeten Stücke zurückgeführt werden konnte. Minutiöse Forschungen, die hauptsächlich \index[personen]{Neumayr, Eva (geb. 1968)}Eva Neumayr oblagen, haben zudem zu überraschenden Erkenntnissen über nachträglich vorgenommene Einreihungen in den \glqq Mozart-Nachlass\grqq{} geführt und aufgezeigt, dass bei den Eintragungen in das Repertorium im 19. Jahrhundert nicht immer mit der gebotenen Umsicht vorgegangen wurde. Durch diese irreführenden Einträge sind falsche Annahmen über Vorbesitzer und Erwerbungsdaten gutgläubig in die \index[personen]{Mozart, Wolfgang Amadé (1756--1791)}Mozart-Literatur eingedrungen. Die Zusammenhänge bei der Akquise und Erschließung werden im Einleitungsteil, zu dem auch Armin \index[personen]{Brinzing, Armin (geb. 1963)}Brinzing und Till \index[personen]{Reininghaus, Till (geb. 1979)}Reininghaus fachkundig beigetragen haben, rekonstruiert.

Der vorliegende Band ist ein Muster für eine geglückte Zusammenarbeit, zunächst zwischen dem Archiv der Erzdiözese Salzburg und der Internationalen Stiftung Mozarteum, dann aber auch für die Kooperation beider Institutionen mit dem \textit{Répertoire International des Sources Musicales}, in dessen Rahmen die Handschriften seit 2014 erfasst und beschrieben wurden.\bigskip

Salzburg und Stuttgart im März 2021,\bigskip\nopagebreak
\begin{multicols*}{3}
	\flushleft
	Johannes Graulich\\ Verleger\\ Carus Verlag\\\columnbreak
	Ulrich Leisinger\\ Wissenschaftlicher Leiter\\ Internationale Stiftung Mozarteum\\\columnbreak
	Thomas Mitterecker\\ Archivleiter\\ Archiv der Erzdiözese Salzburg
\end{multicols*}

\addchap{Vorwort}

Als Josephine \index[personen]{Baroni-Cavalcabò, Josephine von (1787--1860)}Baroni-Cavalcabò sich entschloss, den Nachlass ihres Freundes, Musiklehrers und Lebensgefährten Franz Xaver \index[personen]{Mozart, Franz Xaver Wolfgang (1791--1844)}Mozart dem \begin{itshape}Dommusikverein und Mozarteum\end{itshape} zu übergeben, war ihre Bedingung, \glqq dass diese Sammlung als eine für sich bestehende; vom Sohne, zu Ehren des Vaters gestiftete, nach seinem ausdrücklichen, mir oft ausgesprochenen Willen ungetrennt für ewige Zeiten aufbewahrt; und so den Freunden, die Antheil, an dem unerreichten Ton=Heros nehmen, gezeigt werden können.\grqq{} Diese Bedingung der ungeteilten Aufbewahrung war spätestens ab der Trennung des \begin{itshape}Dommusikverein und Mozarteums\end{itshape} in Dommusikverein und Internationale Mozartstiftung 1880 nicht mehr zu erfüllen.

Umso erfreulicher ist es, dass sich 2014 die Vertreter jener beiden Institutionen, die in der Nachfolge der beiden Vereine den geteilten Mozart-Nachlass aufbewahren, nämlich Dr.~Ulrich Leisinger (Internationale Stiftung Mozarteum, Wissenschaft)  und Dr.~Thomas Mitterecker (Archiv der Erzdiözese Salzburg) zu einer Kooperation entschlossen, um den sogenannten \glqq Mozart-Nachlass\grqq{} wissenschaftlich bearbeiten zu lassen und zu präsentieren. In der Folge konnte der Bestand mit Mitteln der Digitalen Mozart Edition (einem Kooperationsprojekt zwischen The Packard Humanities Institute, Los Altos/CA und der Internationalen Stiftung Mozarteum) und der Erzdiözese Salzburg für das Internationale Quellenlexikon für Musik (\textit{Répertoire Internationale des Sources Musicales}, RISM) bzw. für den Katalog des Österreichischen Bibliothekenverbunds (OBV) katalogisiert werden. Armin \index[personen]{Brinzing, Armin (geb. 1963)}Brinzing, der Leiter der Bibliotheca Mozartiana, übernahm die Aufnahme der autographen Fragmente Wolfgang Amadé Mozarts. Die weiteren Handschriften wurden von Eva Neumayr und Lars \index[personen]{Laubhold, Lars (geb. 1972)}Laubhold (der das Projekt bereits 2015 verließ) katalogisiert, die Bücher und musikalischen Drucke von Tobias \index[personen]{Apelt, Tobias (geb. 1974)}Apelt (der sich 2018 aus dem Projekt zurückzog) bearbeitet.

Der nun vorliegende Katalog wurde ursprünglich von Stephan Hirsch (RISM-Zentralredaktion) als alphabetischer Katalog der Handschriften aus der RISM-Datenbank generiert. Um das Verständnis für die Geschichte des Materials zu erleichtern, schien es jedoch geboten, nach der ursprünglichen Ordnung, die sich im alten Katalog der Bibliothek des Dommusikverein und Mozarteums, dem \begin{itshape}Repertorium über die musikalische Bibliothek des Dom=Musik=Vereines u. Mozarteum’s zu Salzburg\end{itshape}\footnote{A-Sd, o. Sign.} erhalten hat, vorzugehen. Hierzu waren zahlreiche Arbeitsschritte nötig:
\begin{itemize}
	\item Die musikalischen Drucke und Bücher mussten aus dem Katalog des Österreichischen Bibliothekenverbunds herausgezogen und in den Katalog integriert werden.\footnote{Dazu wurden sie von Tobias \index[personen]{Apelt, Tobias (geb. 1974)}Apelt in eine Excel-Liste eingetragen, die Eva Neumayr bearbeitet und in das LaTeX-Dokument integriert hat.}
	\item Jede einzelne Quelle musste dem entsprechenden \textit{Repertoriums}-Eintrag zugeordnet werden. Nach fehlende Quellen wurde gesucht, ein etwaiger Verlust kommentiert.
	\item Alle Personen, Titel, Textanfänge, Werke und Wasserzeichen wurden mit einem Index-Eintrag versehen.
	\item Alle englischen Ausdrücke -- in der RISM-Datenbank sind Einordnungstitel, Schlagwörter, Wasserzeichen etc. in englischer Sprache eingetragen --  wurden ins Deutsche übersetzt.
	\item Die Beschreibungen der Wasserzeichen wurden übersetzt und indexiert. In der Handschriftenaufnahme wird auf die Wasserzeichen nur noch durch eine Nummer verwiesen. Eine Beschreibung des jeweiligen Wasserzeichens findet sich in der Wasserzeichenliste ab Seite \pageref{Wz}. 
	\end{itemize} 
Der nun vorliegende Katalog ist also ein historischer: Die Signaturen sind nach der alten Ordnung gereiht. Um sicherzustellen, dass trotzdem Titelanfänge, Werke, Namen, Wasserzeichen etc. auffindbar bleiben, gibt es am Ende des Katalogs ausführliche Indizes, die die Recherche ermöglichen und erleichtern.

Mit dem gedruckten Katalog \textit{Der \glqq Mozart-Nachlass\grqq{}. Musikalien aus dem Besitz der Söhne W. A. Mozarts in Salzburg} ist indes nur ein Teil des ursprünglichen Vorhabens realisiert. Ziel bleibt, den gesamten Nachlass in einem Portal virtuell der interessierten Öffentlichkeit zur Verfügung zu stellen. Wenn schon die Bedingung der \glqq ungetrennten Aufbewahrung\grqq{}, unter der Josephine \index[personen]{Baroni-Cavalcabò, Josephine von (1787--1860)}Baroni-Cavalcabò die Sammlung dem \begin{itshape}Dommusikverein und Mozarteum\end{itshape} 1844 übergab, heute nicht mehr erfüllt werden kann, so wird diese gemeinsame, virtuelle Präsentation des Nachlasses die Bedingung Josephine \index[personen]{Baroni-Cavalcabò, Josephine von (1787--1860)}Baroni-Cavalcabòs in zeitgemäßer Form einlösen.

Die Herausgeberin dankt ihren Kollegen Dr.~Armin \index[personen]{Brinzing, Armin (geb. 1963)}Brinzing (Fragmente), Dr.~Lars \index[personen]{Laubhold, Lars (geb. 1972)}Laubhold (Handschriften) und Dr.~Tobias \index[personen]{Apelt, Tobias (geb. 1974)}Apelt (Drucke), auf deren Aufnahmen in den Datenbanken von RISM und des OBV der Katalog teilweise basiert, Dr.~Ioana \index[personen]{Geanta, Ioana (geb. 1982)}Geanta, Dr.~Ulrich \index[personen]{Leisinger, Ulrich (geb. 1964)}Leisinger, Dr.~Till Reininghaus und Mag. Thomas Karl \index[personen]{Schmid, Thomas Karl (geb. 1974)}Schmid für das Korrekturlesen sowie allen genannten für  viele anregende Diskussionen. Dr.~Armin \index[personen]{Brinzing, Armin (geb. 1963)}Brinzing hat als Leiter der Bibliotheca Mozartiana die dort befindlichen Quellen für das Projekt zur Verfügung gestellt und in jeder Weise gefördert. Vor allem gilt mein Dank dem wissenschaftlichen Leiter der Internationalen Stiftung Mozarteum, Dr.~Ulrich Leisinger, und dem Leiter des Archivs der Erzdiözese, Dr.~Thomas Mitterecker, die das Projekt ermöglicht, begleitet und unterstützt haben.\bigskip

Salzburg, im März 2021,\bigskip


Eva Neumayr
	  